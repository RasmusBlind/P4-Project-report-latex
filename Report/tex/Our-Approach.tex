In this chapter we present our approach of implementing an automatic transcription of beatboxing. The first subsection will argue for the delimitation of the chosen beatboxing sound classes to be transcribed. Secondly, a section about data collection will describe how we have gathered data both for use in the implementation and in the evaluation of the system's performance. The actual transcription of the system will then be described. This is divided into three parts; how the system will segment audio in order to analyse it, the analysis of the segmentation, i.e. feature extraction and classification, and finally a presentation of an application with a user interface.

All implementation of the system is done through MATLAB\footnote{\url{http://www.mathworks.se/products/matlab/}} using its built-in scripting system. This allows for easy analysis and processing of digital signals.


\section{Choosing Sound Classes}
Based on some of the previous works by \cite{Stowell2010} and \cite{QBBB} a set of three basic beatboxing sounds have been chosen for this experiment. The first of them is the kick drum, which is also known as the bass drum because of its deep-sounding nature. The spectrum of the kick drum lies in the lower frequencies as seen in figure \ref{fig:kick-wave}. The second sound is the snare drum and we have been focusing on the k-snare as opposed to the p-snare, which is performed with an initial k-sound \footnote{Psh-sound according to SBN, see section \ref{SBN}}. Lastly the hi-hat cymbal was chosen, which has a frequency spectrum a bit higher than the snare as illustrated in figure \ref{fig:snare-wave} and \ref{fig:hihat-wave}.

\begin{figure}[h]
	\centering
	\begin{subfigure}[b]{0.275\textwidth}
		\includegraphics[width=\textwidth]{fig/Kick-wave.png}
		\caption{Kick drum}
		\label{fig:kick-wave}
	\end{subfigure}
	\begin{subfigure}[b]{0.275\textwidth}
		\includegraphics[width=\textwidth]{fig/Snare-wave.png}
		\caption{Snare drum}
		\label{fig:snare-wave}
	\end{subfigure}
	\begin{subfigure}[b]{0.35\textwidth}
		\includegraphics[width=\textwidth]{fig/Hihat-wave.png}
		\caption{Hi-hat}
		\label{fig:hihat-wave}
	\end{subfigure}
	\caption{A presentation of 3 beatboxing sounds' waveform (top) and spectrogram
	\label{fig:chosen-sounds} (bottom).}
\end{figure}

What we can see from these waveforms and spectra is that the kick drum differs significantly in its frequency content in relation to the snare and hi-hat. The snare and hi-hat do initially seem to have a clear difference in their frequencies, however they do not seem to lie in a very concentrated area of the spectrum, indicating that they might prove difficult to distinct from each other based on the frequency content.
\section{Data Collection}
\label{sec:data-collecting}
In order to build a system to transcribe beatboxing by amateurs we had to gather a dataset based on this aspect. 
Non-probability sampling was used to collect the data, in which 19 people participated, and it took place in the main building of Aalborg University Copenhagen. 

The participants were placed in a chair surrounded by small partition walls (see cfigure \ref{data-collection-pic}) to limit noise from the surroundings. They were asked to produce 5-10 of each of the three sounds, and also to perform a self-improvised short mix of the sounds.

To record the beatboxing sequences we used a e815-S dynamic cardioid microphone attached to an H4n portable recorder. The sampling rate was 96 kHz and 24 bit precision.

The gathered data was collected on one track used as a database for our system. In order to utilize this database the sound file had to be manually annotated. This means that for each sound its onset was denoted and saved, and the type of sound labelled according to the three sound classes. This was done using Sonic Visualiser\footnote{\url{http://www.sonicvisualiser.org/}}, which generated a text file with the needed information.

\begin{figure}[h]
	\begin{center}
		\includegraphics[height=5cm]{fig/dataset_collection.JPG}
		\caption{Data collection booth.}
		\label{data-collection-pic}
	\end{center}
\end{figure}
\section{Segmentation}
This section will go through how the segmentation of the sound was achieved. The segmentation is needed to locate each sound's endpoints in a signal, i.e. the positions in time at which each sound starts and ends. This is a necessary part of the transcription system as we need to be able to distinct the sounds from each other and not the whole signal as a combination of many sounds.

In this transcription system the segmentation is made by calculating the logarithm of the RMS of each window and if this value, from one window to the next, goes above a threshold it is considered as the starting point of a sound. Similarly when the value falls below the threshold the end of the sound is registered.

The MMATLAB function for segmenting the signal analyses the signal using a specified window size and window skip. When all endpoints in the signal are found, a cell array is returned containing the signal divided into segments so that each cell contains the frames for that segment.

\begin{figure}[h]
	\begin{center}
		\includegraphics[scale =  0.4]{fig/SegmentationPic.png}
		\caption{An example of how the signal will be segmented. On the top the signal is plotted in the time domain with vertical lines indicating the segmentation points (endpoints). The bottom shows the logarithm of the RMS to the signal from above.}
		\label{SegmentationPic}
	\end{center}
\end{figure}

An example of the result from this segmentation is shown in figure \ref{SegmentationPic}. At the top we see the input signal and vertical lines indicating the endpoints of each segment. On the bottom of the figure a graph of the change in the logarithm of the RMS is plotted where we see a peak at each segment in the signal.
\section{Features}
All features should be usable independently of each other, thus each calculation of a feature is implemented in its own function that will return one or more values about that feature. In the following we have divided the description of the features in 3 categories: the time domain features, spectral domain features and the MFCC in its own category.

% TIME DOMAIN SUBSEC ---- START
\subsection{Time Domain Features}
The time domain features consisting of the Root Mean Square (RMS) and Zero-Crossings (ZC) are implemented as functions that each takes the signal or just a segment as input. The RMS function will return a value indicating what the mean energy of the input is, while the total number of zero-crossings will be returned from the ZC function. Programmatically the implementation is rather simple, as can be seen in figure \ref{snippet-RMS} and \ref{snippet-ZC}, where a single loop iterates through all input samples to calculate the output.

\begin{lstlisting}[caption=Matlab implementation of the RMS algorithm., label=snippet-RMS]
function rms = P4_RMS(x)
    % rms = P4_RMS(x)
    %
    % @param x: signal vector.
    % @retval: returns the root mean square of the signal x.
    rms = 0;
    for ii = 1:length(x)
       rms = rms + x(ii)^2; 
    end
    rms = sqrt(rms / length(x));
end
\end{lstlisting}

\begin{lstlisting}[caption=Matlab implementation of the ZC algorithm., label=snippet-ZC]
function zc = P4_zero_crossing(signal)
    % zc = P4_zero_crossing(x)
    %
    % @param x: signal vector.
    % @retval Returns the total zero crossings of the signal x.
    zc = 0;
    for ii = 2:length(signal)
        if (signal(ii) * signal(ii-1) < 0)
            zc = zc + 1;
        end
    end
end
\end{lstlisting}
% TIME DOMAIN SUBSEC ---- END

% SPECTRAL DOMAIN SUBSEC ---- START
\subsection{Spectral Domain Features}
Spectral domain features uses the information gathered in a spectrogram of the sound. Before being able to actually do any calculations of the spectral features, the spectrogram needs to be produced. Using the built-in function in Matlab for calculating the Fast Fourier Transform (FFT) we can produce a spectrogram of the input sound/segment by looping over the input using the specified window size and skip, see appendix \ref{app:feat-spectrogram}. This spectrogram will then be forwarded to the function corresponding to each of the spectral feature-calculations implemented as follows.

The spectral features implemented are: (a) Centroid, (b) Flux, (c) Rolloff and (d) Skewness, all of which are implemented using the code provided by Alexander Lerch\footnote{\url{http://www.audiocontentanalysis.org/code/}} \citep{ACA}. The extracted value from each of these features is a mean value of the whole signal segment sent into the function. This value is directly used in the classifier to identify and describe that specific segment. See appendices under \ref{app:features} for complete implementation.
% SPECTRAL DOMAIN SUBSEC ---- END

\subsection{Mel Frequency Cepstral Coefficients}
Th
\section{Classification using Nearest Neighbour}
This chapter will give information on the classification method called k-Nearest Neighbour (k-NN).

The NN is shortly "given a collection of data point and query points in an m-dimentional metric space, find the data point that is closets to the query point"
\citep{meaningfulNN}.

This means that the NN classification will need a dataset that can train a system \citep{Sinyor05}, the data used for training has to be annotated so that the program that makes the classification knows what the different data represent. 
Once the data has been acquired and trained, it will need to be measured against the value-description. This is because difference in pronunciation and method can make it difficult for the computer to differentiate. A solution to this is to use make use of different features to describe the variation.

When a test-sound is to be classified, a feature vector is calculated using the same features and feature parameters as used when training the k-NN.
The euclidean distance is then calculated between the test-feature vector and all the train-feature vectors \citep{NNHD}.
The k will then be how many neighbours that has to be compared at before determining what the new input is, then the most represented sound will be chosen and labelled as the class of the input sound \citep{introKNN}.

\begin{figure}[h]
	\begin{center}
		\includegraphics[scale = 0.5]{fig/KNNfig.jpg}
		\caption{Illustration of how k-NN will divide the space up with two different classes \citep{introKNN}}
		\label{KNN fig}
	\end{center}
\end{figure}

\colorbox{red}{OUR IMPLEMENTATION WILL BE DESCRIBED HERE (STILL NEEDED!!)}
\section{Matlab Application}
Things to remember for the application:
\begin{itemize}
	\item Information on plots
	\begin{itemize}
		\item Axes titles
		\item Axes labels
	\end{itemize}
	
\end{itemize}

With all the mathematics behind the whole segmentation, feature extraction and classification parts, we realized that instead of having to manually manage each and every function, a more intuitive way of utilizing the whole setup was needed. This lead to the production of a graphical user interface (GUI) consisting of different elements the user could interact with in order to take a piece of sound, segment it and then get an automatic classification. On figure \ref{app-flowchart} a flowchart is provided, presenting the general flow of the application.

\begin{figure}
\caption{Some presentation of the flow of the application}
\label{app-flowchart}
\end{figure}


\paragraph{Application Workflow} \hspace{0pt} \\
asdfkjasdl sldjf asdkfj klsja flsajkdflkasj dfksajd flksajdf lksjad flkasjd flkasjd flksajdf lskajd flkasj dflkasj dflkjasdlfk aslkdfj lksa dfjk sdfj alskdjf lk sdlkf alks jdf lksjdflk sd fj sdf sdf ajsdlfkj salfk jlsdk flk.
\section{Beatboxing as an artform}
Beatboxing as an artform originated from the hip hop culture back in the 1980s. It was derived by vocally imitating drum-machines known as beatboxes e.g. the Roland TR808, which was often used in the production of hip hop music, as a way to create grooves and beats using the mouth, nose and throat \citep{proctor2012}. The self-proclaimed pioneer of beatboxing, Doug E. Fresh also known as the "the Original Human Beatbox" was a key player in promoting beatboxing as an artform. Other mentionable artists were Darren Robinson (Fat Boys), Biz Markie and Rahzel, who also shaped and improved the techniques used in beatboxing \citep{Hess2007}.

While beatboxing primarily involves the imitation of percussive sounds, other beatboxers attempts to imitate bass, guitar or other effects. An example of this is Michael Winslow, an American comedian, actor and beatboxer, probably best known for his ability to make sound effects e.g. imitation of phones or helicopters with his voice in the movie Police Academy from 1984\footnote{\url{www.imdb.com/title/tt0087928/?ref_=fn_al_tt_1}}. 

Through time beatboxing has evolved to become a complex instrumental expression and an artform which is constantly advancing and adapting to modern instruments and audio technologies \citep{proctor2012}. Take for example the beatboxer Tom Thum who authentically imitates the pounding techno-rhythm from within a night club at a TEDx conference \citep{TEDx}. With improved techniques and sophisticated microphone technology beatboxing has become an instrumental element in many music genres besides hip hop, and has as an example been used in pop-music by The King of Pop Michael Jackson\footnote{\url{https://www.youtube.com/watch?v=LQuKbzWHvIo}}.

On an international scale the internet has contributed to the growing popularity of beatboxing, especially with the establishment of the Human Beat Box-community by Alex Tew, who subsequently created the first ever beatboxing convention\footnote{\url{www.beatboxconvention.com}} in 2003; a convention which succeeded in assembling beatboxers from all over the world\footnote{\url{www.humanbeatbox.com}}.
The community is also behind the first beatboxing tutorials that were published in 2003. Furthermore Gavin Tyte and Mark Splinter developed the Standard Beatbox Notation (SBN) as a simple and consistent method to describe sounds and rhythms used in human beatboxing \citep{Tyte}.
\section{ Introduction }
\subsection{ Imitation of Instruments }
In the modern ages people have been imitating instruments for many reasons. One could be the need to describe or invent melodies, rhytms, tones and compositions, and being able to describe them to others. Another reason could be the lack of instruments or perhaps is tapping and humming simply the most intuitive ways to imitate rythms and melodies. Nevertheless is the act of vocalizing percussive sounds probably as old as music itself \citep{Sinyor05}.
\subsection{ Vocal Percussion }
Using mouth, lips and throat to produce percussion sounds and effects is seen in many cultures \citep{Sinyor05}. North American scat singing, mainly seen in jazz, by applying rythmic elements to singing nonsense syllables referred to as "onomatopeia", I.e. by Ella Fitzgerald and Louis Armstrong \citep{Janer_syllablingon}. Vocal imitation of percussion also comes in humorous variations as in the southern American predecessor to beatboxing called "Eefing", described as a "a kind of hiccupping, rhythmic wheeze". Known artists as Joe Perkins hit the charts in 1963 with the song "Little Eeefin' Annie" featuring Jimmie Riddle, who was acknowledged for his skills in the genre at the time \citep{jennifersharpe2006}.

 The voice has also been employed to imitate instruments for pedagogic purposes. For example to teach Cuban percussion, "Vayttari" Indian music, Peking opera and Japanese Noh flute \citep{Janer_syllablingon}.

The modern day equivalent is known as Beatboxing which is primarily linked to the early hip-hop-culture in the 1980s. The self-proclaimed pioneer of Beatboxing, Doug E. Fresh also known as ‘The Human Beatbox’ was a key player in making Beatboxing famous. However, other earlier artists have been known to use similar techniques of vocal percussion in published material e.g. Paul McCartney with the song “That Would Be Something” from 1969 \citep{Sinyor05}.
\subsection{ Beatboxing }
Beatboxing is generally limited to percussive sounds produced by the vocal cord and body (e.g. clapping) to imitate rhythms of a drumset i.e. snare drum, hi-hat, kick drums, cymbals etc. but some Beatboxers also imitates bass and guitar and occasionally combined with singing. An example of this is Michael Winslow, an American comedian, actor, and beatboxer probably best known for his ability to make sound effects with his voice in the movie Police Academy from 1984. Beatboxing as an artform was an outspring of the hip hop culture since the 1980s. It was shaped by musical technologies in context with its age, and through time it evolved to become a complex instrumental expression. In the origin of beatboxing it was meant to imitate grooves and beats but soon it utilized sounds like basslines, scratching, effects, noise, and almost every musical instrument and “filters”. –With improved techniques and sophisticated microphone technology beatboxing became a modern instrumental element in many music genres of today.

\subsection{ Motivation }
The motivation for this project is the admiration of innovative art forms and to explore the opportunities that arise from them. It is the fact that any analog instrument can be treated as a source for digital and electronic processing. Paradoxically beatboxing comes from imitating other analog instruments, but every other analog instrument is also victimized by digitalization. Jimi Hendrix pushed the creative scope of playing guitar, Jean Michel Jarre by playing piano and electronic drums. Why should beatboxing not be treated as an analog instrument as well? 
\subsection{ Why Beatboxing? }
Beatboxing is constantly advancing and adapting to modern instruments and audio technologies. Take for example the beatboxer Tom Thum who authentically imitates the pounding techno-rhythm from within a night club at a TEDx conference.\footnote{The beatboxer Tom Thum presents his skills at TEDx Sydney, 2013 - \url{http://tedxsydney.com/site/item.cfm?item=91A3E4E1C290F6C97E32B9F2623C8967}}. Furthermore, he concludes his presentation by using a Kaoss Pad\footnote{The Kaoss Pad is a touchpad effect processor manufactured by Korg - \url{http://en.wikipedia.org/wiki/Kaoss_Pad}} as a loop station (see chapter \ref{loopstation}), by using the playback to generate an imitation of a jazz band with multiple instruments and lead singer. The internet has contributed to the growing popularity of beatboxing, especially with the creation of the Human Beat Box-community\footnote{\url{www.humanbeatbox.com}} by Alex Tew, who subsequently created the first ever Beatboxing convention in 2003\footnote{The International Human Beatbox Convention - \url{http://www.beatboxconvention.com/}}. A convention that succeeded in assembling beatboxers from around the world. The community is also behind the first beatboxing-tutorials, that were made available in 2003. Standard Beatbox Notation (SBN) was developed by Mark Splinter and Gavin Tyte to create a simple and consistent method to aid novice beatboxers, through the use of standard English characters \citep{tyte2005}.

%Not sure if this section is relevant?
It is intuitively the fast way forward to learn grooves and rhythms. It only requires creativity to imitate sounds and instruments, while many people also tends to have the ability of humming with decent outcome. If technology can provide additional possibilities to facilitate the creation music that sounds good, it can be an entertaining way to learn the principles of rhythm, and to produce satisfactory music, even that one does not know the basics or the techniques of a traditional instrument. 

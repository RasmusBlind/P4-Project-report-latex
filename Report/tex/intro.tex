\section{ Introduction }
Beatboxing is, basically, a tradition of vocal imitation of a percussionist playing a wide range of instruments to create a rhythmic and musical groove \citep{Stowell2008}. Beatboxing originated in the 1980s, and is linked to the early hip-hop culture; it is also referred to as “the fifth element of hip-hop”. The self-proclaimed pioneer of Beatboxing, Doug E. Fresh also known as ‘The Human Beatbox’ was a key player in making Beatboxing famous \citep{Hess2007}. However, other earlier artists have been known to use similar techniques of vocal percussion in published material e.g. Paul McCartney with the song “That Would Be Something” from 1969\footnote{\url{http://www.youtube.com/watch?v=VHTCWY7Lvpk&feature=kp}}, where he imitates a drum beat midway into the song. Beatboxing may be performed a capella or with amplifiying means\citep{Stowell2008}. The main difference between beatboxing and other types of vocal percussion such as jazz scat singing, is that beatboxing attempts to disguise the vocal origin of the sound\citep{Stowell2008}, in other words, making the authenticity of the imitation a key component of the success criterion. \citep{Hazan2005}


\subsection{ Beatboxing }
Beatboxing is generally limited to percussive sounds produced by the vocal cord and body (e.g. clapping) to imitate rhythms of a drumset i.e. snare drum, hi-hat, kick drums, cymbals etc. but some Beatboxers also imitates bass, guitar, sound effects and other elements of a traditional composition. An example of this is Michael Winslow, an American comedian, actor, and beatboxer probably best known for his ability to make sound effects, e.g. imitation of phones, helicopters etc. with his voice in the movie Police Academy from 1984 \footnote{\url{www.imdb.com/title/tt0087928/?ref_=fn_al_tt_1}}. Beatboxing as an artform was an outspring of the hip hop culture since the 1980s \citep{Sinyor05}. It was shaped by musical technologies in context with its age, and through time it evolved to become a complex instrumental expression \footnote{\url{http://www.smithsonianmag.com/science-nature/beatboxing-as-seen-through-scientific-images-8558817/?no-ist}}. In the origin of beatboxing it was meant to imitate grooves and beats but soon it utilized sounds like basslines, scratching, effects, noise, and almost every musical instrument and “filters” \citep{proctor2012}. With improved techniques and sophisticated microphone technology beatboxing became a modern instrumental element in many music genres of today. \citep{Benetos2012}

%\subsection{ Software to transcribe instruments }
%Since “modern beatboxing” evolved, it seems like technology is reclaiming it is technological importance. Some artists are beginning to transcribe instruments, adding filters and effects to beatboxing which takes the complexity of beatboxing even further. I.e. Dub FX, Benjamin Stanford from Australia, who samples sequences of vocal basslines, grooves and adds effects to them. Lately he re-recorded ‘Love Someone’ using Roland RC-505, a so called ‘Loop Station’, based on sequential recording of each track in instrumental compositions. [2] This way, he can perform on streets as one person, sounding as an orchestra. ** NEED TO WRITE MORE ON TRANSCRiPTION**

\textit{n continuation}

\subsection{ Motivation }
Beatboxing is constantly advancing and adapting to modern instruments and audio technologies. Take for example the beatboxer Tom Thum who authentically imitates the pounding techno-rhythm from within a night club at a TEDx conference.\footnote{The beatboxer Tom Thum presents his skills at TEDx Sydney, 2013 - \url{http://tedxsydney.com/site/item.cfm?item=91A3E4E1C290F6C97E32B9F2623C8967}}. Furthermore, he concludes his presentation by using a Kaoss Pad\footnote{The Kaoss Pad is a touchpad effect processor manufactured by Korg - \url{http://en.wikipedia.org/wiki/Kaoss_Pad}} as a loop station (see chapter \ref{loopstation}), by using the playback to generate an imitation of a jazz band with multiple instruments and lead singer. The internet has contributed to the growing popularity of beatboxing, especially with the creation of the Human Beat Box-community\footnote{\url{www.humanbeatbox.com}} by Alex Tew, who subsequently created the first ever Beatboxing convention in 2003\footnote{The International Human Beatbox Convention - \url{http://www.beatboxconvention.com/}}. A convention that succeeded in assembling beatboxers from around the world. The community is also behind the first beatboxing-tutorials, that were made available in 2003. Standard Beatbox Notation (SBN) was developed by Gavin Tyte and Mark Splinter to create a simple and consistent method to aid novice beatboxers\citep{tyte2005}, through the use of percussive modelling synthesis with textual words \citep{McLean2009}. This method is also known as vocables \citep{McLean2009}.

Vocables are intuitively a fast way to learn grooves and rhythms that will ultimately aid the novice beatboxer to improve his skills. However, SBN lacks a way in which the beatboxer can determine whether or not he succesfully used the right vocables to transcribe the corresponding percussive instruments. Maybe we can figure out a way to automatically transcribe beatboxing such that a visual or audio system can  generate  the feedback required so the beatboxer can detect errors and omissions, and thus improve his or her skills. The problem statement would thus be:\\
\begin{center}
\textit{“How can beatboxing be transcribed automatically?”}
\end{center}

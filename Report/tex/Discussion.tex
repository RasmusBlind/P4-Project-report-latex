\paragraph{Test results} \hspace{0pt} \\


\paragraph{Bias} \hspace{0pt} \\


\paragraph{Future Prospects} \hspace{0pt} \\
The way the system is implemented now, we can still see that it lacks a bit of improvement... samsing samsing samsing

The level of transcription in our system is based on classifying the signal to indicate what kind of sound is being input. To expand on the transcription part of the system a possibility could be to synthesize a output instead of a text note. This synthesis could generate a sound, representing the sound input by the user, to imitate a musical instrument, e.g. a real kick-/base drum or hi-hat.

Another way of synthesizing an output could be to use a system like earGram in conjunction with our classification\footnote{Homepage for the earGram application: \url{https://sites.google.com/site/eargram/}}. This would make it possible to input a beatboxing sound and then, through concatenative synthesis in earGram, an output sound. Through this concatenative synthesis, the input sound would undergo an analysis to find sound snippets from a database that sound similar to each piece of the input. The output could then sound like real instruments if the database consisted of recordings of real instruments.

Right now only three beatboxing sounds can, to some extend, be classified limiting the user to only be able to use kicks, snares, and hi-hat sounds. If further development on the system should occur, an expansion on the classifiable sounds would be a suitable subject as it would open up for a broader spectrum of sounds/beats to be transcribed.

Another limit of the system is that it is based on beatboxing by amateurs, i.e. the collected dataset is a recording of people that have not genuinely been beatboxing before. To broaden the target group, gathering data from more professionally oriented beatboxers would give more possibilities. This could include a system used to learn how to beatbox. Say, a person who has never been beatboxing before suddenly wants to learn how to master the art, this person could then input a beatboxing sequence into the system. The system would then tell if the person hit the right sound or not, to help on improving the mastering of beatboxing.

	- Be able to classify more sounds
		- Research BB styles (Victor)
	- Improve data collection (Victor)


\paragraph{Where can we use this?} \hspace{0pt} \\


	- Roskilde (alike)
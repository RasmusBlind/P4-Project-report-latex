\section{Other vocal music practices}

Besides hiphop, vocal percussion techniques are seen in many cultures \citep{Sinyor05}. E.g. “scat singing”, mainly seen in jazz, by applying rhythmic nonsense syllables to singing, also referred to as "onomatopeia". I.e. by Ella Fitzgerald and Louis Armstrong \citep{Janer_syllablingon}. Vocal percussion is also seen in many other contexts, both practically, artistically and pedagogical as described briefly in this chapter.
\subsection{ Artistic Purposes}
Vocal imitation of percussion also comes in humorous variations as in the southern American predecessor to beatboxing called "Eefing", described as "a kind of hiccupping, rhythmic wheeze". Known artists as Joe Perkins hit the charts in 1963 with the song "Little Eeefin' Annie" featuring Jimmie Riddle, who was acknowledged for his skills in the genre at the time \citep{jennifersharpe2006}. In northern India recitation of solkattu \footnote{\url{ www.youtube.com/watch?v=d1yN96ZDGm8}} known as “Konnakol” \citep{proctor2012}. Celtic lilting and diddling \footnote{\url{www.youtube.com/watch?v=Rm_oaqW_qRM&list=PLUgNSFkRKerlocdy-qiC2Z879JlH-FfHe}} also have the characteristics of percussive elements \citep{proctor2012}.
\subsection{ Pedagogic Purposes}
Vocal emulation has also been employed to imitate instruments for pedagogic purposes. For example to teach Cuban percussion, "Vayttari" Indian music, Peking opera and Japanese Noh flute \citep{Janer_syllablingon}. For drum notations “changgo” is used in Korea, as vocables for samul nori drumming, whereas Cuban conga players vocalize motifs as “guauganco or tumbao patterns” \citep{proctor2012}.
\subsection{ Compositional Roles }
There are many examples of vocal emulation of instruments as some of them are mentioned above. Beatboxing, which origins from imitation of drum machines e.g. the Roland TR-808 \citep{proctor2012}, imitates the drum track and is therefore adaptable to many types of compositions. Both as a solo element and as an accompanying instrumental element. Compared to other vocal artforms, in particular the artforms mentioned in this chapter, beatboxing can be said to be the most dynamic and evolving form of vocal percussion as it is based on imitation of all modern instruments. Compared to its predecessor “eefing”, beatboxing is less dominating in the rhythmic composition, where many of the other mentioned vocal artforms are also relatively dominant, and are preferably used as solo elements in compositions. Lastly, beatboxing is a relatively new performance style \citep{Stowell2008}, but it is still widely embraced in mainstream music as in the underground hiphop environment. Beatboxing imitates both humorous elements and pure a capella expressions, which makes it interesting to utilize in many contexts. 
As it is widely utilized in the mainstream music industry, and known by some generations in modern ages, it becomes interesting to develop even further or reinvent certain technical aspects for the performing artists and providers of music technology. Which can also be said to be the background and motivation of the further studies.
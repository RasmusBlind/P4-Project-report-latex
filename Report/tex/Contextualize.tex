\section{Other Vocal Music Practices}
Besides hip hop, rhythmic vocal practices are seen in many cultures \citep{Sinyor05} e.g. “scat singing” which is rhythmic nonsense syllables to singing, used by Ella Fitzgerald and Louis Armstrong \citep{Janer_syllablingon}. Vocal music practices is also seen in many other contexts, such as practical, artistical and pedagogical purposes as described briefly in this chapter.
\subsection{ Artistic Purposes}
Vocal imitation of percussion also comes in humorous variations as in the southern American predecessor to beatboxing called "eefing", described as "a kind of hiccupping, rhythmic wheeze" \citep{jennifersharpe2006}. Among known artists was Joe Perkins who hit the charts in 1963 with the song \textit{Little Eeefin' Annie} featuring Jimmie Riddle, and was acknowledged for his skills in the genre at the time  \citep{jennifersharpe2006}. Other examples are the northern India recitation of solkattu \footnote{\url{ www.youtube.com/watch?v=d1yN96ZDGm8}} known as “Konnakol” \citep{proctor2012}, and Celtic lilting and diddling\footnote{\url{www.youtube.com/watch?v=Rm_oaqW_qRM&list=PLUgNSFkRKerlocdy-qiC2Z879JlH-FfHe}} also have the characteristics of percussive elements \citep{proctor2012}.	
\subsection{ Pedagogic Purposes}
Vocal emulation of instruments has also been employed for pedagogic purposes, e.g. to teach Cuban percussion, Indian music \textit{Vayttari}, Peking opera, and Japanese Noh flute \citep{Janer_syllablingon}. For drum notations \textit{changgo} is used in Korea, as vocables for samul nori drumming, whereas Cuban conga players vocalize motifs as “guauganco or tumbao patterns” \citep{proctor2012}.
\subsection{ Compositional Roles }
There are many examples of vocal emulation of instruments where some of them are mentioned above. Beatboxing imitates the drum track and is therefore adaptable to many types of compositions, both as a solo element and as an accompanying instrumental element. Compared to other vocal artforms, in particular the artforms mentioned in this chapter, beatboxing can be said to be a very dynamic and evolving form of vocal percussion as it is based on imitation of all modern instruments. Compared to its predecessor “eefing”, beatboxing seems less dominating in the rhythmic composition, where many of the other mentioned vocal artforms are also relatively dominant, and are preferably used as solo elements in compositions. Beatboxing is a relatively new performance style \citep{Stowell2008}, but is still widely embraced in mainstream music as well as in the underground hip hop environment. Beatboxing imitates both humorous elements and pure a capella expressions, which makes it interesting to utilize in many contexts.
\section{Features}
\label{app:features}
% SPECTROGRAM ---- START
\subsection{Spectrogram}
\label{app:feat-spectrogram}

\begin{lstlisting}[caption=Matlab implementation of calculating the spectrogram., label=snippet-spectrogram]
function spec = P4_Spectrogram(signal, fs, winSize, winSkip)
    % spec = P4_Spectrogram(signal, fs, winSize, winSkip)
    % Calculates the mean spectral centroid of a signal
    %
    % @param signal:    signal vector
    % @param fs:        sampling rate of signal X
    % @param winSize:   analysis window size in seconds
    % @param winSkip:   amount to shift each window in seconds
    
    % Make it a column vector
    if isrow(signal)
        signal = signal';
    end
    
    win_size = winSize * fs;
    win_skip = winSkip * fs;
    wins = 1 + floor((length(signal)-win_size)/win_skip);
    
    L       = win_size;             % Length of each window
    NFFT    = 2^nextpow2(L);        % Length of the fft
    spec    = zeros(NFFT/2+1, wins);% Allocating mem. to spectrogram
    
    for ii = 1:wins
        start = floor((ii-1) * win_skip) + 1;
        stop = floor((ii-1) * win_skip + win_size);
        window = signal(start:stop,1);
        
        Y = fft(window,NFFT)/L;
        spec(:,ii) = 2*abs(Y(1:NFFT/2+1));
    end
end
\end{lstlisting}
% SPECTROGRAM ---- END

% CENTROID ---- START
\subsection{Spectral Centroid}
\label{app:feat-centroid}

\begin{lstlisting}[caption=Matlab implementation of the spectral centroid., label=snippet-specflux] 
function [sct] = P4_MeanSpectralCentroid(signal, fs, winSize, winSkip)
    % sct = P4_SpectralCentroid(X, fs)
    % Calculates the mean spectral centroid of a signal
    %
    % @param X:         signal vector
    % @param fs:        sampling rate of signal X
    % @param winSize:   analysis window size in seconds
    % @param winSkip:   amount to shift each window in seconds
    
    spec = P4_Spectrogram(signal, fs, winSize, winSkip);
    sct = mean(SpectralCentroid(spec, fs));
end

function vsc = SpectralCentroid(X, fs)
% Credit: Alexander Lerch, An Introduction to Audio Content Analysis.
% ===================================================================
% @brief computes the spectral centroid from the (squared) magnitude spectrum
%
% @param X: spectrogram (dimension FFTLength X Observations)
% @param f_s: sample rate of audio data 
%
% @retval v spectral centroid (in Hz)
    X       = X.^2;
    vsc     = ([0:size(X,1)-1]*X)./sum(X,1);
 
    % avoid NaN for silence frames
    vsc (sum(X,1) == 0) = 0;
 
    % convert from index to Hz
    vsc     = vsc / size(X,1) * fs/2;
end
\end{lstlisting}
% CENTROID ---- END

% FLUX ---- START
\subsection{Spectral Flux}
\label{app:feat-flux}

\begin{lstlisting}[caption=Matlab implementation of the spectral flux., label=snippet-specflux]
function [sflx] = P4_MeanSpectralFlux(signal, fs, winSize, winSkip)
    % sct = P4_SpectralCentroid(X, fs)
    % Calculates the mean spectral centroid of a signal
    %
    % @param X:         signal vector
    % @param fs:        sampling rate of signal X
    % @param winSize:   analysis window size in seconds
    % @param winSkip:   amount to shift each window in seconds
    
   spec = P4_Spectrogram(signal, fs, winSize, winSkip);
   sflx = mean(SpectralFlux(spec));
end

function [vsf] = SpectralFlux (X)
% Credit: Alexander Lerch, An Introduction to Audio Content Analysis.
% ===================================================================
% @brief computes the spectral flux from the magnitude spectrum
% called by ::ComputeFeature
%
% @param X: spectrogram (dimension FFTLength X Observations)
%
% @retval v spectral flux

    % difference spectrum (set first diff to zero)
    afDeltaX    = diff([X(:,1), X],1,2);
 
    % flux
    vsf         = sqrt(sum(afDeltaX.^2))/size(X,1);
end
\end{lstlisting}
% FLUX ---- END

% ROLLOFF ---- START
\subsection{Spectral Rolloff}
\label{app:feat-rolloff}

\begin{lstlisting}[caption=Matlab implementation of the spectral rolloff., label=snippet-specrolloff]
function sroff = P4_MeanSpectralRolloff(signal, fs, winSize, winSkip)
    spec = P4_Spectrogram(signal, fs, winSize, winSkip);
    sroff = mean(SpectralRolloff(spec, fs));
end

function [vsr] = SpectralRolloff (X, fs, kappa)
% Credit: Alexander Lerch, An Introduction to Audio Content Analysis.
% ===================================================================
% @brief computes the spectral rolloff from the magnitude spectrum
%
% @param X: spectrogram (dimension FFTLength X Observations)
% @param fs: sample rate of audio data 
%
% @retval v spectral rolloff (in Hz)

    % initialize parameters
    if (nargin < 3)
        kappa   = 0.85;
    end
 
    % allocate memory
    vsr     = zeros(1,size(X,2));
 
    %compute rolloff
    afSum   = sum(X,1);
    for (n = 1:length(vsr))
        vsr(n)  = find(cumsum(X(:,n)) >= kappa*afSum(n), 1); 
    end
 
    % convert from index to Hz
    vsr     = vsr / size(X,1) * fs/2;
end
\end{lstlisting}
% ROLLOFF ---- END

% SKEWNESS ---- START
\subsection{Spectral Skewness}
\label{app:feat-skewness}

\begin{lstlisting}[caption=Matlab implementation of the spectral skewness., label=snippet-specskewness]
function sskw = P4_MeanSpectralSkewness(signal, fs, winSize, winSkip)
    % sskw = P4_SpectralSkewness(X, fs)
    % Calculates the spectral skewness of a signal
    %
    % @param X:         signal vector
    % @param fs:        sampling rate of signal X
    % @param winSize:   analysis window size in seconds
    % @param winSkip:   amount to shift each window in seconds
    
    spec = P4_Spectrogram(signal, fs, winSize, winSkip);
    sskw = mean(SpectralSkewness(spec));
end

function [vssk] = SpectralSkewness (X)
% Credit: Alexander Lerch, An Introduction to Audio Content Analysis.
% ===================================================================
% @brief computes the spectral skewness from the magnitude spectrum
% called by ::ComputeFeature
%
% @param X: spectrogram (dimension FFTLength X Observations)
%
% @retval v spectral skewness
    UseBookDefinition = true;
 
    if (UseBookDefinition)
        % compute mean and standard deviation
        mu_x    = mean(abs(X), 1);
        std_x   = std(abs(X), 1);
 
        % compute skewness
        X       = X - repmat(mu_x, size(X,1), 1);
        vssk    = sum ((X.^3)./(repmat(std_x, size(X,1), 1).^3*size(X,1)));
    else
        % interpret the spectrum as pdf, not as signal
        f       = linspace(0, fs/2, size(X,1));
        % compute mean and standard deviation
        mu_X    = (f * X) ./ (sum(X,1));
        tmp     = repmat(f, size(X,2),1) - repmat(mu_X, size(X,1),1)';
        var_X   = diag (tmp.^2 * X) ./ (sum(X,1)'*size(X,1));
 
        vssk    = diag (tmp.^3 * X) ./ (var_X.^(3/2) .* sum(X,1)'*size(X,1));
    end
 
    % avoid NaN for silence frames
    vssk (sum(X,1) == 0) = 0;
end
\end{lstlisting}
% SKEWNESS ---- END
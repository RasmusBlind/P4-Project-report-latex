\section{Features}
In this section we will describe different kinds of features. The features that will be focused on are the ones that have been presented in the articles reviewed in section \ref{sec:PW} Previous Work.

\subsection{Root Mean Square}
The Root Mean Square (RMS) is a measure of the energy of a signal. The RMS depends on the amplitudes contained in the signal.


The RMS is found by doing these three steps:
(1) First take the square of the amplitude, (2) then take the mean of the squared results, (3) lastly take the square root of the results from the results of part (2) \citep{Bird2007}. \\ Here is the mathematical formula for RMS of a discrete signal \citep{ACA}:
\begin{equation}\label{eq:RMS formular}
\upsilon_{RMS}(n) = \sqrt{\frac{1}{\kappa}\sum_{i=i_s(n)}^{i_e(n)} x(i)^2}
\end{equation}
RMS tells us about the amount of energy used in the signal.

\subsection{Zero Crossing and Zero Crossing Rate}
ZC is a term normally used in image processing, electronics and mathematics \citep{AlZhrani2010}. The ZC is a point on the zero axis, where a signals waveform have crossed from negative to positive (or the other way around), or simply as the name implies where a signal crosses zero.{AlZhrani2010}
In other words ZC is a way of measuring the period of a periodic signal\citep{RWW2012}.


When measuring a frequency it is wise to measure an amount of periods instead of just one period. By having more periods one can reduce errors caused by phase noise \citep{RWW2012}.
Another variant of ZC is called Zero Crossing Rate. ZCR is defined as the amount of times the audio waveform crosses the zero axis \citep{DSShete}.
Below is zero crossing rate formula \citep{ACA} 
\begin{equation}\label{eq:ZCR}
\upsilon_{ZC}(n)= \frac{1}{2 \kappa}\sum_{i=i_s(n)}^{i_e (n)}|sign[x(i)]-sign[x(i-1)]|
\end{equation}

When the sign goes from positive to negative some changes occur in the signal. These changes have an effect on the assumed content of the frequency \citep{ACA}. Depending on how many changes occur e.g. the less the signal changes its sign, the less the frequency is assumed to be in the signal \citep{ACA}.

To sum up, the ZCR can be seen as a measure of high and low frequency signal, since it measures the amount of times the signal crosses the x-axis.

\subsection{Mel-Frequency Cepstral Coefficients}
This section will explain the Mel-Frequency Cepstral Coefficients (MFCC).
The MFCC feature is a compact description of the spectral envelope, and is often used in speech recognition and have been useful in musical processing as well \citep{ACA}.


In audio signal classification a small subset of the resulting MFCCs will contain the principle information, in most cases between 4-20 MFCC are used. The way that the MFCC are calculated are similar to the way humans perceive sound. Instead of a linear frequency scale it uses a non-linear frequency scale (Mel scale) based on the human perception. The MFCC uses Discrete Cosine Transform (DCT) instead of Discrete Fourier Transform (DFT)\citep{ACA}.
The calculation of the $j^{th}$ coefficient $v^j$  MFCC (n) is seen below \citep{ACA}\\
\begin{equation}\label{ eq:MFCC calculation}
  \upsilon ^j  _{MFCC} (n) = \sum_{k'=1}^{\kappa'} log(\vert X' (k',n) \vert)\cos(j(k' - \frac{1}{2})\frac{\pi}{\kappa'})
\end{equation}
\\
There are different ways to implement the MFCC. The main difference between the implementations are the way that the spectrum is calculated. Examples of implementations are HTK \citep{htkbook} found in the HMM tool kit software,  and the implementation found in Slaney's Auditory Toolbox (SAT) \citep{Slaney}.
\\

\subsection{Spectral Centroid}
This section will show what the feature spectral centroid is.\\
The spectral centroid feature will calculate the Center of Gravity (COG) of a spectrum. It is defined by the frequency weight power spectrum normalized by the unweighed sum \citep{ACA}:
\begin{equation}\label{Spectral Centroid eq}
	\upsilon_{SC}(n) = \frac{\displaystyle\sum_{k = 0}^{\frac{\kappa}{2-1}} k\vert X(k,n) \vert^2}{\displaystyle\sum_{k = 0}^{\frac{\kappa}{2-1}} \vert X(k,n) \vert^2 }    
\end{equation} 

The Spectral Centroid can also be calculated using the magnitude spectrum instead of the power spectrum, which means the power is not taken in the calculation \citep{ACA}.
\\
The point found by the Spectral Centroid feature should correlate with the timbre dimension of how sharp or bright the sound is \citep{ACA}. 

\subsection{Spectral Spread}
Spectral Spread measures the power spectrum around the spectral centroid.\\ 
It can be seen as taking the standard deviation of the power spectrum around the spectral centroid, and can be calculated with the following formula \citep{ACA}:
\begin{equation}
	\upsilon_{SS}(n)=\sqrt{\frac{\displaystyle\sum_{k = 0}^{\kappa/2-1}(k-\upsilon_{SS}(n))^2\vert X(k,n)\vert^2}{\displaystyle\sum_{k = 0}^{\kappa/2-1}\vert X(k,n)\vert^2}}
\end{equation}
As the Spectral Centroid can be calculated in different ways, the spread follows accordantly, which means that the spectral spread can also be calculated using the magnitude spectrum instead of the power spectrum \citep{ACA}.

\subsection{Spectral Rolloff}
The Spectral Rolloff is defined as the frequency bin at which the magnitude of the Short Time Fourier Transform (STFT) reaches a percentage K of the overall sum of magnitudes, can be calculated with the following formula \citep{ACA}\\
\begin{equation}\label{ eq:normal spectral rolloff}
	\upsilon_{SR}(n) = i \vert _{\displaystyle\sum_{k = 0}^i \vert X(k, n) \vert = K  \displaystyle\sum_{k = 0}^ {\frac{\kappa}{2-1}}\vert X(k, n) \vert}
\end{equation}
\\
Normal the value for K percentage is around 0.85 or 0.95. Low results indicates insufficient magnitudes components at high frequencies and a low audio bandwidth\citep{ACA}.\\


\subsection{Spectral Flux}
The Spectral Flux is a value for how much the spectrum shape change between the different frames. It can be defined as \citep{ACA}:
\begin{equation}\label{Spectral Flux eq}
	\upsilon_{SF}(n) = \frac{\sqrt{\displaystyle\sum_{k=0}^{\kappa/2-1}(\vert X(k,n)\vert-\vert X(k,n-1)\vert)^2}}{\kappa/2}
\end{equation} 
The Spectral Flux feature can be described as a representation of the roughness of a sound. The result that one will get from a spectral flux feature is in the range from 0 to A, where A is the maximum magnitude possible in the spectrum \citep{ACA}. When looking at Spectral Flux in a signal it will be flat at silence and spike at pitch changes \citep{ACA}.
\section{Spectral Analysis}
In this chapter different way of analysing the spectrum will be covered. Also how it might be useful in the implementation of the transcription system will be considered.
\subsection{Spectral Centroid}
This chapter will show what the feature spectral centroid is.\\
The spectral centroid feature will calculate the center of gravity(COG) of a spectrum it is defined by the frequency weight power spectrum normalized by the unweigth sum \citep{ACA}:
\begin{equation}\label{Spectral Centroid eq}
	\upsilon_{SC}(n) = \frac{\displaystyle\sum_{k = 0}^{\frac{\kappa}{2-1}} k\vert X(k,n) \vert^2}{\displaystyle\sum_{k = 0}^{\frac{\kappa}{2-1}} \vert X(k,n) \vert^2 }    
\end{equation} 
However the spectral centroid can also be calculated using the magnitude spectrum instead of the power spectrum, with means the power is not taken in the calculation\citep{ACA}.
\\
The point found by the spectral centroid feature should correlate with the timbre dimension of how sharp or bright the sound is \citep{ACA}. 

This feature might be used in a transcription system because it should describe how sharp or bright a sound will sound, so this feature could be used to classify as sound. but there might be problems in the pauses/noise areas as seen on figure \ref{Spectral centroid pic} as there is a high peak at some of the pauses. 
\subsection{Spectral Rolloff}
This chapter will explain what the spectral rolloff is.\\
The Spectral Rolloff is defined as the frequency bin at which the magnitude of the STFT reaches a percentage K of the overall sum of magnitudes, can be calculated like\citep{ACA}\\
\begin{equation}\label{ eq:normal spectral rolloff}
	\upsilon_{SR}(n) = i \vert _{\displaystyle\sum_{k = 0}^i \vert X(k, n) \vert = K  \displaystyle\sum_{k = 0}^ {\frac{\kappa}{2-1}}\vert X(k, n) \vert}
\end{equation}
\\
Normal the value for K is around 0.85 or 0.95. Low results indicates insufficient magnitudes components at high frequencies and the a low audio bandwidth\citep{ACA}.\\
Different ways to compute the spectral rolloff can be that only parts of the spectral energy is taken into considerations that is done by use an fmin and a fmax\citep{ACA}
\begin{equation}\label{ eq: fmin and fmax spectral rolloff}
	\upsilon_{SR, \Delta f}(n) = i \vert _{\displaystyle\sum_{k = k(f_{min})}^i \vert X(k, n) \vert = K \displaystyle\sum_{k = k_(f_{min})}^ {k(f_{max})}\vert X(k, n) \vert}
\end{equation}
It can also be common to use the power spectrum
\begin{equation}\label{ eq:power spectral rolloff}
	\upsilon_{SR, pow}(n) = i \vert _{\displaystyle\sum_{k = 0}^i \vert X(k, n) \vert^2 = K \displaystyle\sum_{k= 0}^ {\frac{\kappa}{2-1}}\vert X(k, n) \vert^2}
\end{equation}
\\
For our project the spectral rolloff could also be used to determine the sound in the classification, as the spectral rolloff will tell about the roughness of the sound. 

\subsection{Spectral Flux}
This chapter will explain the feature Spectral Flux and its uses.\\
The spectral Flux is how much the spectrum shape change between the different frames it can be defined as\citep{ACA}:
\begin{equation}\label{Spectral Flux eq}
	\upsilon_{SF}(n) = \frac{\sqrt{\displaystyle\sum_{k=0}^{\kappa/2-1}(\vert X(k,n)\vert-\vert X(k,n-1)\vert)^2}}{\kappa/2}
\end{equation} 
The spectral flux feature can be described as a representation of the roughness of a sound. The result that one will get from a spectral flux feature is in the range from 0 to A, where A is the maximum magnitude possible in the spectrum\citep{ACA}. When looking at spectral flux in a signal it will be flat at silence and spike at pitch changes\citep{ACA}.
For use in note onset detection (finding the start of the note) only an increase in the spectral energy is wanted an one can consider using a different way to calculate the spectral flux\citep{ACA}:
\begin{equation}
	\Delta X(k,n) = \vert X(k,n)\vert-\vert X(k,n-1)\vert
\end{equation}
When doing this all negative values has to be set to zero so that only increase is detected\citep{ACA}.
\\
For a transcription system the spectral flux as described can be considered to do the segmentation of a sound signal because it can detect an onset.
\subsection{Spectral Slope}
This chapter will be on explaining what the spectral slop feature is.\\
The spectral slope as the name indicate is a feature that is a measure of the slop in the spectral energy\citep{ACA}. The spectral slope is calculated using a linear regression of the spectral magnitude spectral, the slop is then estimated using this equation\citep{ACA}
\begin{equation}
	\upsilon _{SSI} (n) = \frac{\displaystyle\sum_{k = 0}^{\kappa/2-1}(k - \mu_k)(\vert X(k,n)\vert - \mu_\vert x _\vert)}{\displaystyle\sum_{k = 0}^{\kappa/2-1}(k - \mu_k)^2}
\end{equation}
\begin{equation}
	= \kappa\frac{\displaystyle\sum_{k= 0}^{\kappa/2-1}k\vert X(k,n)\vert - \displaystyle\sum_{k = 0}^{\kappa/2-1}k\displaystyle\sum_{k = 0}^{\kappa/2-1}\vert X(k,n)\vert}{\kappa \displaystyle\sum_{k = 0}^{\kappa/2-1}\kappa^2-(\displaystyle\sum_{k = 0}^{\kappa/2-1}k)^2}
\end{equation}
The result the one can get from these equations is defined by range of the spectral magnitude.
The spectral slope will be high at pauses and get lower with there is a sound based on that the spectral slope could be considered for the segmentation.
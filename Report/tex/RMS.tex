\section{Root Mean Square}
The root mean square value is as John bird tells in Engineering Mathematics \citep{Bird2007} “the square root of the mean value of the squared values of the quantity over an interval".
\\
As Bird mentions in Engineering mathematics "One of the principal applications of RMS is with alternating current and voltage". \citep{Bird2007} The alternating current (a.c.) is defined as a current which has a heating effect resembling the direct current \citep{Bird2007}.
\\
The RMS is found by doing these three steps:
(1) First take the square of the amplitude, (2) then take the mean of the squared results, (3) lastly take the square root of the results from the results of part (2)\citep{Bird2007}. Here is \cite{Bird2007} mathematical formula for RMS:

\begin{equation}\label{eq:RMS formular}
RMSy = \frac{1}{b-a}\int_a^b\mathrm{y}^{2}\,\mathrm{d}x
\end{equation}

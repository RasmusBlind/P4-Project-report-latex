\section{Choosing Sound Classes}
Based on some of the previous works by \cite{Stowell2010} and \cite{QBBB} a set of three basic beatboxing sounds have been chosen for this experiment. The first of them is the kick drum, which is also known as the bass drum because of its deep-sounding nature. The spectrum of the kick drum lies in the lower frequencies as seen in figure \ref{fig:kick-wave}. The second sound is the snare drum and we have been focusing on the k-snare as opposed to the p-snare, which is performed with an initial k-sound \footnote{Psh-sound according to SBN, see section \ref{SBN}}. Lastly the hi-hat cymbal was chosen, which has a frequency spectrum a bit higher than the snare as illustrated in figure \ref{fig:snare-wave} and \ref{fig:hihat-wave}.

\begin{figure}[h]
	\centering
	\begin{subfigure}[b]{0.275\textwidth}
		\includegraphics[width=\textwidth]{fig/Kick-wave.png}
		\caption{Kick drum}
		\label{fig:kick-wave}
	\end{subfigure}
	\begin{subfigure}[b]{0.275\textwidth}
		\includegraphics[width=\textwidth]{fig/Snare-wave.png}
		\caption{Snare drum}
		\label{fig:snare-wave}
	\end{subfigure}
	\begin{subfigure}[b]{0.35\textwidth}
		\includegraphics[width=\textwidth]{fig/Hihat-wave.png}
		\caption{Hi-hat}
		\label{fig:hihat-wave}
	\end{subfigure}
	\caption{A presentation of 3 beatboxing sounds' waveform (top) and spectrogram
	\label{fig:chosen-sounds} (bottom).}
\end{figure}

What we can see from these waveforms and spectra is that the kick drum differs significantly in its frequency content in relation to the snare and hi-hat. The snare and hi-hat do initially seem to have a clear difference in their frequencies, however they do not seem to lie in a very concentrated area of the spectrum, indicating that they might prove difficult to distinct from each other based on the frequency content.
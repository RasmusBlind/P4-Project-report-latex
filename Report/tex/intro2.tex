\section{ Introduction }
Beatboxing is, basically, a tradition of vocal imitation of a percussionist playing a wide range of instruments to create a rhythmic and musical groove \citep{Stowell2008}. Beatboxing originated in the 1980s, and is linked to the early hip-hop culture; it is also referred to as “the fifth element of hip-hop”. The self-proclaimed pioneer of Beatboxing, Doug E. Fresh also known as ‘The Human Beatbox’ was a key player in making Beatboxing famous \citep{Hess2007}. However, other earlier artists have been known to use similar techniques of vocal percussion in published material e.g. Paul McCartney with the song “That Would Be Something” from 1969.
%\footnote{\url{http://www.youtube.com/watch?v=VHTCWY7Lvpk&feature=kp}}. 
Beatboxing may be performed a capella or with amplifiying means\citep{Stowell2008}.\\ The main difference between beatboxing and other types of vocal percussion such as jazz scat singing, is that beatboxing attempts to disguise the vocal origin of the sound\citep{Stowell2008}, in other words, making the authenticity of the imitation a key component of the success criterion.  An example is the beatboxer Tom Thum who authentically imitates the pounding techno-rhythm from within a night club at a TEDx conference.\footnote{The beatboxer Tom Thum presents his skills at TEDx Sydney, 2013 - \url{http://tedxsydney.com/site/item.cfm?item=91A3E4E1C290F6C97E32B9F2623C8967}}. His performance indicates that beatboxing is constantly changing and adapting to modern instrument and audio technologies. Beatboxing uses a wide selection of vocal techniques to achieve authenticity, most notably is the use of inhaled sounds. \citep{Stowell2008} proposes that this technique allows for a continuous flow of rhythm, and furthermore maintains the auditory illusion of authenticity.\\
\\

%[Several examples of beatboxing, is that types of beatboxing, or persons who perform?]

Just like a musician can transcribe music, computer programs have been created to attempt to automatically transcribe a music signal into its notes. The idea of Automatic Music Transcription (AMT) was first proposed in 1977 by audio researchers James A. Moorer, Martin Piszczalski, and Bernard Galler \citep{Scheirer1998}, who considered the theoretical possibility to use computers to analyse a digital recording of music. Quite rightly the idea was executable, and has led to many computer programs who seek to automatically transcribe audio including music e.g. \textit{ Express Scribe Transcription Playback Software } or \textit{ FTW Transcriber }. These applications all share common features i.e. automatic retrieval of musical information and systems that offer an interactive music experience \citep{Benetos2012}. In continuation of this, AMT raises an interesting question: Can a beatboxing performance, likewise, be transcribed automatically? \\
\\
The internet has contributed to the growing popularity of beatboxing, especially with the creation of the Human Beat Box-community\footnote{\url{www.humanbeatbox.com}} by Alex Tew, who subsequently created the first ever Beatboxing convention in 2003\footnote{The International Human Beatbox Convention - \url{http://www.beatboxconvention.com/}}. A convention that succeeded in assembling beatboxers from around the world. The community is also behind the first beatboxing-tutorials, that were made available in 2003. Standard Beatbox Notation (SBN) was developed by Gavin Tyte and Mark Splinter to create a simple and consistent method to aid novice beatboxers\citep{tyte2005}, through the use of percussive modelling synthesis with textual words \citep{McLean2009}. This method is also known as vocables \citep{McLean2009}.

Vocables are intuitively a fast way to learn grooves and rhythms that will ultimately aid the novice beatboxer to improve his skills. However, SBN lacks a way in which the beatboxer can determine whether or not he succesfully used the right vocables to transcribe the corresponding percussive instruments. Maybe a software-based solution to automatically transcribe beatboxing, could be beneficial in alleviating this problem? A visual or audio system could plausibly  generate  the feedback required so the beatboxer can detect errors and omissions, and thus improve his or her skills.\\
\\

Due to the fact that beatboxing is a subcultural phenomenon, there is very little substantial academic work done on the area \citep{Stowell2008}, but we have found that there are atleast some articles out there that describes a possible solution to beatboxing-transcription. Dan Stowell and Mark D. Plumbley mentions in their technical report \textit{Characteristics of the beatboxing vocal style}, that technical beatboxing-inspired software solutions have been developed. Most notably is Amaury Hazan’s voice controlled drum-machine \citep{Hazan2005}, and Elliot Sinyor \textit{et al.}’s \textit{Autonomous Classification Engine (ACE)} that is used to classify beatboxing sounds\citep{Sinyor05}. These software-based solutions approach the problem of beatboxing-transcription in a similar fashion i.e. segmentation and classification. These two articles have been particularly relevant to our project, as it has given insight into the method of transcription. Furthermore, both papers advocate that delimitation is needed in relation to the number of different types of percussive sounds. 
Although Dan Stowell and Mark D. Plumbley claims that neither of the projects are clear on whether they are developed in contact with practicing beatboxers \citep{Stowell2008},  Elliot Sinyor \textit{et al.} clearly describes that their test involved  three beatboxer from an a cappella groups whereas two of them participate in beatboxing competitions.\citep{Sinyor05}. \\
\\
 
In this project, we explore the feasibility of automatic beatbox transcription, through thorough investigation inter alia of MFCC, spectral analysis, endpoint detection and the accompanying features. We design, implement a simple automatic beatbox transcription system as a proof of concept. We conduct several experiments on the system with acquired test statistic distributed between the training and test set in a 70\%/30\% ratio. The data will be evaluated using a Chi squared test to determine the optimal settings to identify the best performance. From these, we discuss how our system can be improved and might be expanded to other types of vocal percussion than beatboxing.\\
\\
This report is structured as follows. In the second section (Background), we review beatboxing as an artform, dicuss the acoustic properties of a chosen beatbox sounds i.e. hi-hat (hh), kick (k) and snare drum (s), furthermore we will contextualize the tradition with some other vocal music practices to bring the project into perspective. In the third section (Methods of beatboxing- transcription), we review other approaches that have been designed for automatic beatboxing transcription. State-of-the-art applications, features and proceedings are scrutinized which will provide insight into classification and transcription methods. In the fourth section (Our Approach), we will present the design and implementation of our developed software solution. In the fifth section (Evaluation), the evaluation of our system will be presented in accordance with the aforementioned method. Section six (Discussion) stands in continuation of the preceding chapter, as we will discuss acquired results. We will, furthermore, in light of the results, suggest avenues for future improvements of our system. In the concluding section, we discuss the success of our system with respect to existing systems, and additional applications outside of transcription.
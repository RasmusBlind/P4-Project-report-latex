Beatboxing is a tradition of vocal imitation of a percussionist playing a wide range of instruments to create a rhythmic and musical groove \citep{Stowell2008}. 
%Beatboxing originated in the 1980s, and is linked to the early hip-hop culture; it is also referred to as “the fifth element of hip-hop”. The self-proclaimed pioneer of Beatboxing, Doug E. Fresh also known as ‘The Human Beatbox’ was a key player in making Beatboxing famous \citep{Hess2007}. Earlier artists have been known to use similar techniques of vocal percussion in published material e.g. Paul McCartney with the song “That Would Be Something” from 1969.
%\footnote{\url{http://www.youtube.com/watch?v=VHTCWY7Lvpk&feature=kp}}. 
Beatboxing may be performed a capella or with amplifiying means \citep{Stowell2008}. \\The internet has contributed to the growing popularity of beatboxing, especially with the creation of the Human Beat Box-community\footnote{\url{www.humanbeatbox.com}} by Alex Tew, who subsequently created the first ever Beatboxing convention in 2003\footnote{The International Human Beatbox Convention - \url{http://www.beatboxconvention.com/}}. A convention that succeeded in assembling beatboxers from around the world. The community is also behind the first beatboxing-tutorials, that were made available in 2003. Many elementary beatboxing technique examples can be found on humanbeatbox.com\footnote{\url{https://www.humanbeatbox.com/forum/content.php/425-sounds}}. Reggie Watts, a currently popular artist, employs beatboxing in many creative ways for music and performances\footnote{\url{https://www.youtube.com/watch?v=CJQU22Ttpwc}}.
%The main difference between beatboxing and other types of vocal percussion such as jazz scat singing, is that beatboxing attempts to disguise the vocal origin of the sound\citep{Stowell2008}, in other words, making the authenticity of the imitation a key component of the success criterion.  An example is the beatboxer Tom Thum who authentically imitates the pounding techno-rhythm from within a night club at a TEDx conference.\footnote{The beatboxer Tom Thum presents his skills at TEDx Sydney, 2013 - \url{http://tedxsydney.com/site/item.cfm?item=91A3E4E1C290F6C97E32B9F2623C8967}}. His performance indicates that beatboxing is constantly changing and adapting to modern instrument and audio technologies. Beatboxing uses a wide selection of vocal techniques to achieve authenticity, most notably is the use of inhaled sounds. \citep{Stowell2008} proposes that this technique allows for a continuous flow of rhythm, and furthermore maintains the auditory illusion of authenticity.\\

%[Several examples of beatboxing, is that types of beatboxing, or persons who perform?]

Just like a musician can transcribe music, computer programs have been created to attempt to automatically transcribe a music signal into its notes. The idea of Automatic Music Transcription (AMT) was first proposed in 1977 by audio researchers James A. Moorer, Martin Piszczalski, and Bernard Galler \citep{Scheirer1998}, who considered the theoretical possibility to use computers to analyse a digital recording of music. Quite rightly the idea was executable, and has led to many computer programs who seek to automatically transcribe audio including music e.g. \textit{ Express Scribe Transcription Playback Software } or \textit{ FTW Transcriber }. 
Automatic transcription of audio has shown its use in areas as diverse as biology\footnote{\url{http://www.birds.cornell.edu/brp/raven/ravenoverview.html}} and video games\footnote{\url{https://www.singstar.com/}}.
While transciption of audio sources has proven to be useful in many areas, interesting questions and aspects arise, if beatboxing could also be transcribed, and possibly used as MIDI controllers, instrumental karaoke, pedagogical purposes, and many more.
%These applications all share common features i.e. automatic retrieval of musical information and systems that offer an interactive music experience \citep{Benetos2012}. In continuation of this, AMT raises an interesting question: Can a beatboxing performance, likewise, be transcribed automatically? \\

% Standard Beatbox Notation (SBN) was developed by Gavin Tyte and Mark Splinter to create a simple and consistent method to aid novice beatboxers\citep{tyte2005}, through the use of percussive modelling synthesis with textual words \citep{McLean2009}. This method is also known as vocables \citep{McLean2009}.

%Vocables are intuitively a fast way to learn grooves and rhythms that will ultimately aid the novice beatboxer to improve his skills. However, SBN lacks a way in which the beatboxer can determine whether or not he succesfully used the right vocables to transcribe the corresponding percussive instruments. Maybe a software-based solution to automatically transcribe beatboxing, could be beneficial in alleviating this problem? A visual or audio system could plausibly  generate  the feedback required so the beatboxer can detect errors and omissions, and thus improve his or her skills.\\


Due to the fact that beatboxing is a subcultural phenomenon, there is very little substantial academic work done on the area \citep{Stowell2008}, but we have found several articles that propose possible solutions to beatboxing-transcription. Amaury Hazan created a beatbox controlled drum-machine \citep{Hazan2005}. Elliot Sinyor \textit{et al.} used the \textit{Autonomous Classification Engine (ACE)} to classify beatboxing sounds from a dataset he created with the help of 3 accomplished beatboxers and 3 non-beatboxers \citep{Sinyor05}. They further found that reducing the amount of classes to 3, they obtained a 98.15\% accuracy, versus the initial best result 95.55\%. Dan Stowell and Mark B. Plumbley approached the transcription with the goal of reaching real-time speeds \citep{Stowell2008}, by investigating delayed decision-making with various features. They also created a beatboxing sound dataset for the purpose, with the help of users at humanbeatbox.com\footnote{Available at: \url{https://archive.org/details/beatboxset1}}. Together, the three papers touch upon a well of relevant knowledge on areas such as dataset creation, audio segmentation, audio features,  classification, and system performance.

%Both papers advocate that delimitation is needed in relation to the number of different types of percussive sounds, ie. sound classes. 
%Although Dan Stowell and Mark D. Plumbley claims that neither of the projects are clear on whether they are developed in contact with practicing beatboxers \citep{Stowell2008},  Elliot Sinyor \textit{et al.} clearly describes that their test involved three beatboxers from an a cappella groups whereas two of them participate in beatboxing competitions.\citep{Sinyor05}. \\

In our project, we investigate automatic transcription of beatboxing. We create a dataset based on recordings of amateur beatboxers, which we test upon. We implement and evaluate audio features and classification, testing their performance with multiple varying parameters.
Furthermore we create an interface for easy analysis and classification of sounds (stored or recorded). Finally we use the results and experiences to discuss how our systems performance can be improved, and to further develop a useful tool for audio analysis.

%In this project, we explore the feasibility of automatic beatbox transcription, through thorough investigation inter alia of MFCC, spectral analysis, endpoint detection and the accompanying features. We design, implement a simple automatic beatbox transcription system as a proof of concept. We conduct several experiments on the system with acquired test statistic distributed between the training and test set in a 70\%/30\% ratio. The data will be evaluated using a Chi squared test to determine the optimal settings to identify the best performance. From these, we discuss how our system can be improved and might be expanded to other types of vocal percussion than beatboxing.\\

The structure of this report is as follows: In the second section (Background), beatboxing is reviewed as an artform, the acoustic properties of chosen beatbox sounds i.e. hi-hat (hh), kick (k) and snare drum (s) are briefly discussed. Furthermore the tradition of other vocal music practices is contextualized with beatboxing to perspectivate the project. The third section (Methods of beatboxing-transcription) contains reviews of other approaches that have been designed for automatic transcription of beatboxing. State-of-the-art applications, features and proceedings are considered to provide insight into classification and transcription methods. In the fourth section (Our Approach), the design and implementation of our developed software solution are presented. In the fifth section (Evaluation), the evaluation of our system is presented in accordance with the aforementioned method. The sixth section (Discussion) stands in continuation of the preceding chapter, as the acquired results are discussed. Furthermore, in light of the results, avenues for future improvements of our system are suggested. The success of our system with respect to existing systems, and additional applications outside of transcription are discussed in the concluding section.
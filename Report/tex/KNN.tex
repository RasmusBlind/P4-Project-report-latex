\section{KNN}
this chapter will give information on the classification method called KNN (K nearest neighbour).\\
the KNN classification will need a dataset often refereed to as the training data, then it needs some different values of the data from the training data, for that one can use different features. When the different object of data has been determined the different data they also needs to be set into different classes, in case of beatboxing sound these classes could be a kick sound, a snare etc. then the data can be view as being in at different places from the value that they got from the features with the different classes. now when there comes a new input, the input can be given a value from the features and then based on what classes that is around the input it can be classified based on its nearest neighbour. the K in KNN is how many of the closest neighbours that are taken into consideration. 
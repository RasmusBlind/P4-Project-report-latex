\section{Onset Detection}
In short an onset is the start of a sound event \citep{ACA}. 
An important thing to know about onset segmentation is the difference between transients, onset and attack \citep{Bello2005}.
Attack of a note is the time during amplitude changes, from the beginning of a sound to the maximum amplitude is reached \citep{ACA}.
Transient also starts when the attack starts, at the beginning of a sound\citep{Bello2005}. When the note reaches its quasi-periodic state, the transient is considered to end \citep{ACA}.
Onset is a chosen instance which marks the temporarily extended transient \citep{Bello2005}. 


There are at lot of different ways to find the onset for an audio signal. It depends on which kind of audio signal one has \citep{Bello2005}. 
Lerch refers to three different onset times: Note Onset Time (NOT), Acoustic Onset Time (AOT) and Perceptual Onset Time (POT).


NOT is "the time when the instrument is triggered to make a sound [...] this is not necessarily the time when the signal becomes detectable or audible" \citep{ACA}.
\\
AOT is "the first time when a signal or an acoustic event is theoretically measurable. Sometimes the AOT is called physical onset time" \citep{ACA}.
\\
POT is "the first time when the event can be perceived by the
listener" \citep{ACA}.


Out of these three one must find the most suitable onset detection type. As Lerch stated NOT is not necessarily detectable in audio signals \cite{ACA}. Furthermore  Lerch states that POT is often the most suitable onset time for onset detection. \citep{ACA}.

In our project we can use this segmentation to see which part of the audio file have sound, and where there is no sound. 	
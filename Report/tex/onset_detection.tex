\section{onset detection}
in short an onset is the start of a sound event (Lerch 2012 source XX)
its important to know the difference between: transients onset and attack. 
Attack: attack of a note is the time during amplitude changes, from the beginning to the maximum amplitude is reached(lerch 2012 source XX )
Transient: Transient also starts when the attack starts, at the beginning of of a sound. it ends when the note reaches its quasi-periodic state. (Lerch 2012 source XX )
Onset: Onset is a chosen instant which marks the temporally extended transient.(Bello et al. 2005 XX)mostly that interval is as mentioned earlier, the beginning of a sound. 
\\

There are at lot of different ways of  finding onset detection for a audio signals, it depends on which kind of audio signal you have.
In: An Introduction to Audio Content Analysis, Lerch refers to three different onset times, namely: Note Onset Time (NOT), Acoustic Onset Time (AOT) and Perceptual Onset Time 


\section{onset detection}
In short an onset is the start of a sound event \cite{ACA} 
its important to know the difference between: transients onset and attack. 
Attack: attack of a note is the time during amplitude changes, from the beginning to the maximum amplitude is reached \cite{ACA}
Transient: Transient also starts when the attack starts, at the beginning of of a sound. it ends when the note reaches its quasi-periodic state. \cite{ACA}
Onset: Onset is a chosen instant which marks the temporally extended transient. \cite{Bello2005} Mostly that interval is as mentioned earlier, the beginning of a sound. 
\\
There are at lot of different ways of  finding onset detection for a audio signals, it depends on which kind of audio signal you have.
In: An Introduction to Audio Content Analysis, \cite{ACA} refers to three different onset times, namely: Note Onset Time (NOT), Acoustic Onset Time (AOT) and Perceptual Onset Time 
\\
NOT: "the time when the instrument is triggered to make a sound." \cite{ACA}
\\
AOT: "the first time when a signal or an acoustic event is theoretically measurable. Sometimes the AOT is calledphysical onset time." \cite{ACA}
\\
POT: "the first time when the event can be perceived by the
listener." \cite{ACA}
\\
In \cite{ACA} Lerch compare a lot of different results from different papers dealing with onset detection. He concludes that the onset perception differs depending on the test data and that the "deviations evoked evoked by motoric abilities seem to be in the same range." \cite{ACA}
\\

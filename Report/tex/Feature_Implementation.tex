\section{Features}
All features should be usable independently of each other, thus each calculation of a feature is implemented in its own function that will return one or more values about that feature. In the following we have divided the description of the features in 3 categories: the time domain features, spectral domain features and the MFCC in its own category.

% TIME DOMAIN SUBSEC ---- START
\subsection{Time Domain Features}
The time domain features consisting of the Root Mean Square (RMS) and Zero-Crossings (ZC) are implemented as functions that each takes the signal or just a segment as input. The RMS function will return a value indicating what the mean energy of the input is, while the total number of zero-crossings will be returned from the ZC function. Programmatically the implementation is rather simple, as can be seen in figure \ref{snippet-RMS} and \ref{snippet-ZC}, where a single loop iterates through all input samples to calculate the output.

\begin{lstlisting}[caption=Matlab implementation of the RMS algorithm., label=snippet-RMS]
function rms = P4_RMS(x)
    % rms = P4_RMS(x)
    %
    % @param x: signal vector.
    % @retval: returns the root mean square of the signal x.
    rms = 0;
    for ii = 1:length(x)
       rms = rms + x(ii)^2; 
    end
    rms = sqrt(rms / length(x));
end
\end{lstlisting}

\begin{lstlisting}[caption=Matlab implementation of the ZC algorithm., label=snippet-ZC]
function zc = P4_zero_crossing(signal)
    % zc = P4_zero_crossing(x)
    %
    % @param x: signal vector.
    % @retval Returns the total zero crossings of the signal x.
    zc = 0;
    for ii = 2:length(signal)
        if (signal(ii) * signal(ii-1) < 0)
            zc = zc + 1;
        end
    end
end
\end{lstlisting}
% TIME DOMAIN SUBSEC ---- END

% SPECTRAL DOMAIN SUBSEC ---- START
\subsection{Spectral Domain Features}
Spectral domain features uses the information gathered in a spectrogram of the sound. Before being able to actually do any calculations of the spectral features, the spectrogram needs to be produced. Using the built-in function in Matlab for calculating the Fast Fourier Transform (FFT) we can produce a spectrogram of the input sound/segment by looping over the input using the specified window size and skip, see appendix \ref{app:feat-spectrogram}. This spectrogram will then be forwarded to the function corresponding to each of the spectral feature-calculations implemented as follows.

The spectral features implemented are: (a) Centroid, (b) Flux, (c) Rolloff and (d) Skewness, all of which are implemented using the code provided by Alexander Lerch\footnote{\url{http://www.audiocontentanalysis.org/code/}} \citep{ACA}. The extracted value from each of these features is a mean value of the whole signal segment sent into the function. This value is directly used in the classifier to identify and describe that specific segment. See appendices under \ref{app:features} for complete implementation.
% SPECTRAL DOMAIN SUBSEC ---- END

\subsection{Mel Frequency Cepstral Coefficients}
Using the MFCC algorithm provided by Malcolm Slaney we have implemented a version of it that has been adjusted by Bob L. Sturm. This adjustment makes it possible to use different analysis window sizes and skips, please refer to provided Matlab-script "mfcc.m". The algorithm will provide a feature vector with 40 cepstral coefficients that describe every window of the input sound. To use these coefficients, we take a mean over all windows, i.e. for every coefficient we take the mean value of that coefficient over all windows. Out of these 40 mean values, we use the first 20 coefficients.
\section{Features}

\begin{enumerate}
	\item Time domain features
	\begin{enumerate}
		\item RMS
		\item ZC
	\end{enumerate}
	\item Spectral domain features
	\begin{enumerate}
		\item Centroid
		\item Flux
		\item Rolloff
		\item Skewness
	\end{enumerate}
	\item MFCC
\end{enumerate}

The features are each divided into a subscript of the whole application. The time domain features consisting of the Root Mean Square (RMS) and Zero-Crossings (ZC) are implemented as functions that each takes the signal or just a segment as input. The RMS function will return a value indicating what the mean energy of the input is, while the total number of zero-crossings will be returned from the ZC function. Programmatically the implementation is rather simple, as can be seen in figure \ref{snippet-RMS} and \ref{snippet-ZC}, where a single loop iterates through all input samples.

\begin{figure}
\begin{lstlisting}
function rms = P4_RMS(x)
% rms = P4_RMS(x)
%
% @param x: signal vector.
% @retval: returns the root mean square of the signal x.
    rms = 0;
    for ii = 1:length(x)
       rms = rms + x(ii)^2; 
    end
    rms = sqrt(rms / length(x));
end
\end{lstlisting}
\caption{Code snippet. Matlab implementation of the RMS algorithm.}
\label{snippet-RMS}
\end{figure}

\begin{figure}
\begin{lstlisting}
function zc = P4_zero_crossing(signal)
% zc = P4_zero_crossing(x)
%
% @param x: signal vector.
% @retval Returns the total zero crossings of the signal x.
    zc = 0;
    for ii = 2:length(signal)
        if (signal(ii) * signal(ii-1) < 0)
            zc = zc + 1;
        end
    end
end
\end{lstlisting}
\caption{Code snippet. Matlab implementation of the ZC algorithm.}
\label{snippet-ZC}
\end{figure}
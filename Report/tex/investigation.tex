\section{Investigation}
A great amount of effort has been put into investigating possible solutions to how transcription can be achieved through sound and music computing. While many articles, books and papers on the topic has been scrutinized, three specific articles we found to be of utmost importance to the project. The articles addresses relevant topics i.e. transcription - / and sound classification of beatboxing.

\subsection{Transcription}
In the article \textit{Towards Automatic Transcription of Expressive Oral Percussive Performances}, Amoury Hazan propose a solution to automatic transcription, that involves sound segregation, oral percussive descriptors, and machine learning techniques. Sound segregation is described as a simple method that fits well with monophonic oral percussive recordings, which requires few computing resources \citep{Hazan2005a}.	

His project aims to reduce the gap between the user and the device (keyboard, drum pad), among other things in order to offer an aid to those musicians who cannot transcript a beat they have in mind.
Through research of the taxonomy of human phonemes, and assuring perfect percussive events through segregation and transcription, the resulting drum score lack information concerning how the performer has modulated the produced sound. Thus, Hazan argues that energy and resonance frequency variation has to be defined together with effective computational methods to track them.
	
The considerations made by Hazan may serve as a benchmark tool which may suffice to serve as a possible solution to the problem. The technical descriptions and reflections could serve as an inspiration regarding how to construct the software based transcription-solution.

\subsection{Sound Classification}

[INSERT ARTICLE HERE - Beatbox Classification Using ACE \textit{by} \citep{Sinyor05} ]

It is important to understand how standard beatboxing techniques works, in order to build a system that recognizes vocal instrumental sounds. Each specific instrumental sound triggers the system to understand which sounds to replace with synthesized instruments. In this project the focus will be on basic percussion: kick, snare and cymbal.

The sequential process of classification contains 2 fundamental steps[**source**], to be described in this chapter: 

1.	Data Collection

	a)	Recording
	b)	Segmentation
	
3. Classification

	a)	Features
	b)	Feature Selection

Data Collection: 
Initially it is important to collect a range of sounds related to beatboxing. Therefore a dataset must be recorded both from experienced and unexperienced beatboxers. Subjects are instructed to record a small sequence of beatboxing, consisting of kick, snare and cymbal sounds (for this project). All sounds are recorded in 96Khz .wav files. 

** GRAPHICS DESCRIBING BASIC SAMPLES ** 

The recording for this project was done with a Zoom H4N linked to a standard phantom powered studio mic typically used by beatboxers. To get the most useful recordings the aim was to come as close as possible to the beatboxing sound environment, including background noise and the technical framework.


Segmentation:
Once all recordings has been collected

Classification: 





[INSERT ARTICLE HERE - Delayed Decisionmaking in real-time Beatbox Percussion Classification \textit{by} \citep{Stowell2010} ]
